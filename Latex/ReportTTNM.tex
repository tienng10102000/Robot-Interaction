\documentclass[runningheads]{llncs}
%\usepackage{amscd,amsfonts,enumerate}
\usepackage[left=3.00cm, right=2.00cm, top=2.00cm, bottom=2.00cm]{geometry} % căn lề theo quy chuẩn KLTN
\usepackage{color, fancyhdr, graphicx, wrapfig}
\usepackage[unicode]{hyperref}
\usepackage[utf8]{vietnam} % Sử dụng tiếng việt
\usepackage{graphicx} % Cho phép chèn hỉnh ảnh
\usepackage{fancybox} % Tạo khung box
\usepackage{authblk}
\usepackage{indentfirst} % Thụt đầu dòng ở dòng đầu tiên trong đoạn
%\usepackage{amsthm} % Cho phép thêm các môi trường định nghĩa
%\usepackage{latexsym} % Các kí hiệu toán học
%\usepackage{amsmath} % Hỗ trợ một số biểu thức toán học
%\usepackage{amssymb} % Bổ sung thêm kí hiệu về toán học
%\usepackage{amsbsy} % Hỗ trợ các kí hiệu in đậm
\usepackage{times} % Chọn font Time New Romans
\usepackage{array} % Tạo bảng array
\usepackage{enumitem} % Cho phép thay đổi kí hiệu của list
\usepackage{subfiles} % Chèn các file nhỏ, giúp chia các chapter ra nhiều file hơn
%\usepackage{titlesec} % Giúp chỉnh sửa các tiêu đề, đề mục như chương, phần,..
\usepackage{chngcntr} % Dùng để thiết lập lại cách đánh số caption,..
\usepackage{pdflscape} % Đưa các bảng có kích thước đặt theo chiều ngang giấy
\usepackage{afterpage}
\usepackage{capt-of} % Cho phép sử dụng caption lớn đối với landscape page
\usepackage{multirow} % Merge cells
\usepackage{fancyhdr} % Cho phép tùy biến header và footer

\usepackage[nonumberlist, nopostdot, nogroupskip]{glossaries}
\usepackage{glossary-superragged}
\setglossarystyle{superraggedheaderborder}
\usepackage{setspace}
\usepackage{parskip}
\input{DMVT_glossary/DMVT_glossary}

%\usepackage[natbib,backend=biber,style=ieee]{biblatex} % Giúp chèn tài liệu tham khảo

\bibliography{TLTK_bibliography/main.bib} % chèn file chứa danh mục tài liệu tham khảo vào 

\graphicspath{{figures/}{../Hinh_figures/}} % Thư mục chứa các hình ảnh

\counterwithin{figure}{chapter} % Đánh số hình ảnh kèm theo chapter. Ví dụ: Hình 1.1, 1.2,..


\newtheorem{dn}{Định nghĩa}[section]
\newtheorem{tc}[dn]{Tính chất}
\newtheorem{dl}[dn]{Định lí}
\newtheorem{md}[dn]{Mệnh đề}
\newtheorem{bd}[dn]{Bổ đề}
\newtheorem{hq}[dn]{Hệ quả}
\newtheorem{nx}[dn]{Nhận xét}
\newtheorem{vd}{Ví dụ}
\pagenumbering{roman}\pagestyle{plain}
\pagestyle{fancy}
\lhead{\it Tương tác người máy}
\rhead{\it Chẩn đoán bệnh lý ở người}
\lfoot{\it 1CTT18A1}
\rfoot{\it ĐHSPKTVL}
\renewcommand{\headrulewidth}{1,2pt}
\renewcommand{\footrulewidth}{1,2pt} % Cái này là tiêu đề chạy


\begin{document}
	\fontsize{13pt}{18pt}\selectfont % Lệnh thay đổi cỡ chữ thành cỡ 13, cỡ dòng 18 (theo quy chuẩn của Khóa Luận TN).
	\setlength{\baselineskip}{18truept}
	\subfile{Bia_cover/cover_Bia} % Phần bìa
	
	\chapter*{MỞ ĐẦU}
	\addcontentsline{toc}{chapter}{{\bf MỞ ĐẦU}\rm} %Đưa lời nói đầu vào mục lục
	\subfile{MoDau_Abstract_Overview/MoDau_Abstract_Overview}
	
	\tableofcontents % Mục lục
	\newpage
	\listoffigures % Mục lục hình
	\newpage
	\listoftables % Mục lục bảng

	\glsaddall 
	\renewcommand*{\glossaryname}{DANH SÁCH THUẬT NGỮ}
	\renewcommand*{\acronymname}{Danh sách từ viết tắt}
	\renewcommand*{\entryname}{Tiếng Anh}
	\renewcommand*{\descriptionname}{Tiếng Việt}
	\printnoidxglossary
	
	\begin{tabular}{l l }
		
	\end{tabular}
	
	\chapter{CƠ SỞ LÝ THUYẾT} % Chương 1
	\pagenumbering{arabic}            % Danh lai trang
	\subfile{Chuong_chapters/Chuong1_CSLT}
	\newpage
	
	\chapter{CÔNG CỤ THỰC HIỆN} % Chương 2
	\subfile{Chuong_chapters/Chuong2_CongCu}
	\newpage
	
	\chapter{NỘI DUNG NGHIÊN CỨU} % Chương 3
	\subfile{Chuong_chapters/Chuong3_NDNC}
	\newpage

	\chapter{KẾT QUẢ THỰC HIỆN} % Chương 4
	\subfile{Chuong_chapters/Chuong4_KQTH}
	\newpage

	\chapter{KẾT LUẬN} % Chương 5
	\subfile{Chuong_chapters/Chuong5_KL}
	\newpage
	
	
	\chapter*{TÀI LIỆU THAM KHẢO}
	\subfile{TLTK_bibliography/TLTK_biolio}
	
\end{document}